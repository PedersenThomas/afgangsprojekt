\chapter{Analyse}

%Baggrund. Referere til et Link på nettet hvor praktik rapporten ligger.
Projektet er lavet i samarbejde med Adaheads K/S og Responsum K/S. \linebreak
Responsum K/S er et firma der har receptionister klar til at tage opkald for andre firmaer. Til at hjælpe dem med at få informationen omkring det pågældende firma der ringes til, har de et ældre system der ikke længere opdateres af firmaet bag det og koden til det er ikke tilgængelig for dem. Det betyder at de fejl og mangler de har fundet ikke bliver udbedret. Derfor besluttede Responsum K/S at starte Adaheads K/S til at løse deres problem ved at lave et nyt system, men denne gang er koden underlagt en GPL licens og de er derved sikre på at ingen kan løbe med koden.

\begin{figure}[ht!]
\centering
\includegraphics[scale=0.5]{images/adaheads_system.png}
\caption{En oversigt over Adaheads system.}
\label{fig:adaheadssystem}
\end{figure}
Det nye system består af flere dele. En del er programmet Callflow der kommuniker med Freeswitch og holder styr på hvilke kald der er i anlægget og sender forespørgelser til Freeswitch om f.eks. at koble to opkald sammen. Det program som receptionister sidder med er lavet som en web side, der sørger for at holde sig opdateret på opkald via den før nævnte server. For at hente information ud omkring medarbejder, organisationer osv. fra databasen er der lavet små programmer til det også, så receptionisternes klient kan få hentet information.

\section{Problemformulering}
Responsum K/S har de det kalder serviceagenter, som er medarbejdere til at løbende tjekke informationen igennem og holde den opdateret. For at kunne få informationen ind i det nye system skal der være en hjemmeside hvor det kan indtastes. Foruden så skal man også kunne konfigurarer en receptions dialplan. Koden for alle dele af projektet skal være open source og serveren skal kunne køre på Linux eller BSD.

\section{Kravspecifikation}
For at finde ud af hvad der skal laves i et projekt, er det et rigtig godt udgangspunkt at starte med en kravspecifikation.
\begin{enumerate}
  \item Man skal kunne oprette, ændre, deaktiver og aktiver organizationer.
  \item Man skal kunne oprette, ændre, deaktiver og aktiver receptioner.
  \item Man skal kunne oprette, ændre, deaktiver og aktiver kontaktpersoner.
  \item Man skal kunne oprette, og rette en dialplan
  \item En dialplan skal kunne sendes ud til en Freeswitch server, i det rette format.
  \item Man skal kunne oprette og rette telefonnumre.
  \item Man skal kunne oprette, rette, og deaktivere brugere.
  \item Man skal kunne oprette, rette, og fjerne kalender begivenheder.
  \item Man skal kunne rette, og fjerne beskeder.
  \item Man skal kunne få en regning ud for en organization.
\end{enumerate}

\section{Use case}
For at få en fornemmelse for, hvor kravene bliver brugt henne og om man har overset noget er det en god ide, at lave use case\citep{LarmanUml}.
Mange operationer på serveren, er klassiske CRUD (Create, Read, Update, Delete) operationer. Derfor har jeg valgt at undlade nogen usecase, da de ikke variere væsentligt fra nogle af de andre.


\section{Opret reception}
Følgende senariere kan også finde sted for organizationer og kontaktpersoner i receptioner.

\begin{table}[h]
    \begin{tabular}{|p{3cm}|p{8.3cm}|}
    \hline
    Formål         & At oprette en ny reception til en organisation                              \\ \hline
    Succestilstand & Receptionen bliver oprettet og gemt i databasen.                            \\ \hline
    Fejltilstand   & Der bliver ikke oprettet noget receptionen, og bruger 
                     får en fejl meddelelse. \\ \hline
    Aktør          & Serviceagent                                                                \\ \hline
    \end{tabular}
\end{table}

\begin{enumerate}
  \item Login med tilstrækkelig rettigheder.
  \begin{enumerate}
    \item Hvis brugeren ikke har noget login eller ikke har tilstrækkelige rettigheder. Sendes brugeren ud til login vinduet igen.
  \end{enumerate}
  \item Tryk på "ny reception" knappen.
  \item Indtast informationen den nye reception skal have.
  \item Tryk gem. Derved bliver der sendt en forespørgsel til serveren om at oprette en ny reception.
  \begin{enumerate}
    \item Hvis brugeren har indtastet ugyldig informationen eller har mistet forbindelsen til serveren, får brugeren at vide at der er sket en fejl.
  \end{enumerate}
\end{enumerate}

%\section{Ændre reception}
%\begin{enumerate}
%  \item Login med tilstrækkelig rettigheder.
%  \item Vælg den rette reception.
%  \item Tilret information.
%  \item Tryk på gem, og information bliver updateret i systemet.
%\end{enumerate}

%\section{Slet reception}
%\begin{enumerate}
%    \item Login med tilstrækkelig rettigheder.
%    \item vælg den rette reception.
%    \item Tryk på slet, hvilket sletter reception på serveren.
%\end{enumerate}

%\section{Aktivere reception}
%\begin{enumerate}
%    \item Login med tilstrækkelig rettigheder.
%    \item vælg den deaktiveret reception.
%    \item Tryk på aktivere, hvilket aktivere reception på serveren igen.
%\end{enumerate}

\begin{table}[h]
    \begin{tabular}{|p{3cm}|p{8.3cm}|}
    \hline
    Formål         & At ændre en receptions dialplan                              \\ \hline
    Succestilstand & Dialplanen er ændret og ligger klar i databasen, til at compileren henter den.                         \\ \hline
    Fejltilstand   & Der bliver ikke gemt nogen ændringer, og bruger får en fejl meddelelse. \\ \hline
    Aktør          & Serviceagent                                                                \\ \hline
    \end{tabular}
\end{table}

\section{Dialplan}
\begin{enumerate}
    \item Login med tilstrækkelig rettigheder.
  \begin{enumerate}
    \item Hvis brugeren ikke har noget login eller ikke har tilstrækkelige rettigheder. Sendes brugeren ud til login vinduet igen.
  \end{enumerate}
    \item Vælg receptionen.
    \item Begynd og redigere dens dialplan.
    \item Tryk gem og ændringerne bliver sendt til serveren der gemmer det i databasen.
  \begin{enumerate}
    \item Hvis brugeren har indtastet ugyldig informationen eller har mistet forbindelsen til serveren, får brugeren at vide at der er sket en fejl.
  \end{enumerate}
\end{enumerate}