\chapter{Terminologi}

\paragraph{Dialplan} er den opskrift der fortæller hvordan et opkald skal håndteres i en PBX

\paragraph{Freeswitch}
{
  name=Freeswitch,
  description={er en PBX på lige fod med Asterisk.}
}

\paragraph{MVC} (Model View Controller) er en arkitektur der først har modellen, altså typer til at opbevare data så det giver mening i forhold til domænet. Der efter er der view som via det data der kommer fra modellen, at lave en repræsentation af dataen som sendes ud. TIl sidste Controller der forbinder det hele, og som har alt bussiness logikken.

\paragraph{PBX} (Private Branch Exchange) er et andet udtryk for telefonanlæg. Altså det system der tager sig af at håndter opkald

\paragraph{Receptionist} er en medarbejder som tager opkald og sørger for at omstille eller tage imod en besked fra opkalder

\paragraph{Regex} (regular expression) eller på dansk kendt som Regulære udtryk. Er et sprog der kan sammenligne tekster og finde matches eller erstatte dele af teksten

\paragraph{Serviceagent} er en der sørger for at holde data omkring firmaerne opdateret, så receptionisterne har den rigtige information
