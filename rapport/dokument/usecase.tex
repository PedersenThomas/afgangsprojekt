\section{Use case}
For at få en fornemmelse for, hvor kravene bliver brugt henne og om man har overset noget er det en god ide, at lave use case\citep{LarmanUml}.
Mange operationer på serveren, er klassiske CRUD (Create, Read, Update, Delete) operationer. Derfor har jeg valgt at undlade nogen usecase, da de ikke variere væsentligt fra nogle af de andre.


\section{Opret reception}
Følgende senariere kan også finde sted for organizationer og kontaktpersoner i receptioner.

\begin{table}[h]
    \begin{tabular}{|p{3cm}|p{8.3cm}|}
    \hline
    Formål         & At oprette en ny reception til en organisation                              \\ \hline
    Succestilstand & Receptionen bliver oprettet og gemt i databasen.                            \\ \hline
    Fejltilstand   & Der bliver ikke oprettet noget receptionen, og bruger 
                     får en fejl meddelelse. \\ \hline
    Aktør          & Serviceagent                                                                \\ \hline
    \end{tabular}
\end{table}

\begin{enumerate}
  \item Login med tilstrækkelig rettigheder.
  \begin{enumerate}
    \item Hvis brugeren ikke har noget login eller ikke har tilstrækkelige rettigheder. Sendes brugeren ud til login vinduet igen.
  \end{enumerate}
  \item Tryk på "ny reception" knappen.
  \item Indtast informationen den nye reception skal have.
  \item Tryk gem. Derved bliver der sendt en forespørgsel til serveren om at oprette en ny reception.
  \begin{enumerate}
    \item Hvis brugeren har indtastet ugyldig informationen eller har mistet forbindelsen til serveren, får brugeren at vide at der er sket en fejl.
  \end{enumerate}
\end{enumerate}

%\section{Ændre reception}
%\begin{enumerate}
%  \item Login med tilstrækkelig rettigheder.
%  \item Vælg den rette reception.
%  \item Tilret information.
%  \item Tryk på gem, og information bliver updateret i systemet.
%\end{enumerate}

%\section{Slet reception}
%\begin{enumerate}
%    \item Login med tilstrækkelig rettigheder.
%    \item vælg den rette reception.
%    \item Tryk på slet, hvilket sletter reception på serveren.
%\end{enumerate}

%\section{Aktivere reception}
%\begin{enumerate}
%    \item Login med tilstrækkelig rettigheder.
%    \item vælg den deaktiveret reception.
%    \item Tryk på aktivere, hvilket aktivere reception på serveren igen.
%\end{enumerate}

\begin{table}[h]
    \begin{tabular}{|p{3cm}|p{8.3cm}|}
    \hline
    Formål         & At ændre en receptions dialplan                              \\ \hline
    Succestilstand & Dialplanen er ændret og ligger klar i databasen, til at compileren henter den.                         \\ \hline
    Fejltilstand   & Der bliver ikke gemt nogen ændringer, og bruger får en fejl meddelelse. \\ \hline
    Aktør          & Serviceagent                                                                \\ \hline
    \end{tabular}
\end{table}

\section{Dialplan}
\begin{enumerate}
    \item Login med tilstrækkelig rettigheder.
  \begin{enumerate}
    \item Hvis brugeren ikke har noget login eller ikke har tilstrækkelige rettigheder. Sendes brugeren ud til login vinduet igen.
  \end{enumerate}
    \item Vælg receptionen.
    \item Begynd og redigere dens dialplan.
    \item Tryk gem og ændringerne bliver sendt til serveren der gemmer det i databasen.
  \begin{enumerate}
    \item Hvis brugeren har indtastet ugyldig informationen eller har mistet forbindelsen til serveren, får brugeren at vide at der er sket en fejl.
  \end{enumerate}
\end{enumerate}