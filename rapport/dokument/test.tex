\chapter{Test}
Man kan teste sine programmer på mange forskellige måde, man kan blandt andet lave unit test, integration test og acceptance test. Unit test er test af mindre dele af programmet. Meningen med dem er at de er små og hurtige, så når der arbejdes på en lille krog af programmet, så har man flere test til det, som man kører igen og igen for at sikre sig at når man udvikler det at man ikke ødelægger noget på vejen\citep{osherove2010artofunittesting}.
Integretions-test er hvor der testes om hvor vidt et program overholder dens interface. Så her ser man ikke på de små komponenter et program består af men om det overordnet virker som det skal.
Acceptance test er når produktet afleveres til kunden og der undersøges om det lever op til deres krav.

I dette projekt er blevet lavet automatiske integrations test til serveren. Testene blev skrevet i Python, da det har nogle gode test biblioteker og fordi så kunne det nemt intregreres med AdaHeads nuværende test miljø som netop også kører Python. Der blev kun lavet integrations test fordi når man sender en forespørgelse til serveren berører den kun nogle få komponenter, så i praktiske kommer det til at minde lidt om Unit testing.
For at teste om der nu leveres data tilbage i det rette format, så bruges der et værktøj kaldet JSON\_Schema til at validere op i mod. Et set af regler i JSON\_Schema er skrevet i JSON formattet og det bruges så til at tjekke om visse nøgler er tilstede, eller om værdierne overholder et regulære udtryk eller hvis det er et tal kan man validere at værdien ligger imellem et interval.

Jeg havde desværre ikke tid til at sætte test op til klienten og compileren men jeg forestiller mig at hvis jeg havde kunne man teste klienten med et framework som Selenium. Det kan nemlig lave til automatiske test af hjemmesider, så man ikke behøver manuelt at prøve at fremprovokere fejlene. Compileren oversætter til XML så til det kunne man lave en DTD (\textbf{D}ocument \textbf{T}ype \textbf{D}efinition) som er et værktøj til at verificere at ens XML dokument er opbygget efter et set at regler. En anden oplagt måde at teste compileren på, er at se om Freeswitch håndterer kald på den måde man specificeret i kaldplanen man sendte til compileren. Så ved at have nogle software telefoner til at ringe ind, så kan man teste om der sker det ønskede og dermed om compileren oversætter rigtigt. 
