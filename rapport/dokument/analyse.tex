\chapter{Analyse}

%Baggrund. Referere til et Link på nettet hvor praktik rapporten ligger.
Responsum K/S er en virksomhed der besvarer andre virksomheders opkald, som en ekstern reception. For at kunne dette gør de brug af telefonanlæg, databaser og PC baserede klienter. Deres nuværende system er udviklet tilbage i 90'erne og består for en stor del af komponenter der ikke længere udvikles på, og derfor efterhånden slæber rundt på en del fejl og mangler. Som følge heraf gik Adaheads K/S igang med at udvikle et fremtidssikret open source system baseret på GPL licensen for at sikre, at koden altid er tilgængelig for systemets brugere.

Responsum K/S har behov for at kunne lave kaldplaner til at modtage og dirigere opkald til henholdsvis receptionister eller automatiske systemer som IVR menuer og telefonsvarer. Der skal knyttes information til alle indgående opkald, sådan at receptionister og systemer kan håndtere dem korrekt. For at kunne oprette og vedligholde kaldplaner og virksomhedsinformation skal der være administrative værktøjer til rådighed.

\pagebreak

\section{Oversigt}
AdaHeads' produkt består af følgende systemer:
\begin{figure}[ht!]
\centering
\includegraphics[scale=0.5]{images/adaheads_system.png}
\caption{En oversigt over Adaheads system.}
\label{fig:adaheadssystem}
\end{figure}
\begin{enumerate}
	\item{Freeswitch PBX til håndtering af opkald} 
	\item{Kaldplan-compiler til generering af Freeswitch XML kaldplaner}
	\item{Administrativ browser baseret klient til interaktion med REST interfaces}
	\item{Relationel database}
	\item{REST intefaces, fordelt på flere HTTP servere}
	\item{CallFlow kald dirigenten håndterer kaldrelateret kommunikation mellem Freeswitch og resten af systemet}
	\item{Browser baseret klient til receptionister}
\end{enumerate}
Komponenterne markeret med rød er designet og konstrueret i dette projekt
%Linjen skal være på samme side som figuren.

\section{Kravspecifikation}
For at finde ud af hvad der skal laves i et projekt, er det et rigtig godt udgangspunkt at starte med en kravspecifikation.
\begin{enumerate}
  \item Man skal kunne oprette, ændre, deaktiver og aktiver organizationer.
  \item Man skal kunne oprette, ændre, deaktiver og aktiver receptioner.
  \item Man skal kunne oprette, ændre, deaktiver og aktiver kontaktpersoner.
  \item Man skal kunne oprette, og rette en dialplan
  \item En dialplan skal kunne sendes ud til en Freeswitch server, i det rette format.
  \item Man skal kunne oprette og rette telefonnumre.
  \item Man skal kunne oprette, rette, og deaktivere brugere.
  \item Man skal kunne oprette, rette, og fjerne kalender begivenheder.
  \item Man skal kunne rette, og fjerne beskeder.
  \item Man skal kunne få en regning ud for en organization.
\end{enumerate}

\section{Use case}
For at få en fornemmelse for, hvor kravene bliver brugt henne og om man har overset noget er det en god ide, at lave use case\citep{LarmanUml}.
Mange operationer på serveren, er klassiske CRUD (Create, Read, Update, Delete) operationer. Derfor har jeg valgt at undlade nogen usecase, da de ikke variere væsentligt fra nogle af de andre.


\section{Opret reception}
Følgende senariere kan også finde sted for organizationer og kontaktpersoner i receptioner.

\begin{table}[h]
    \begin{tabular}{|p{3cm}|p{8.3cm}|}
    \hline
    Formål         & At oprette en ny reception til en organisation                              \\ \hline
    Succestilstand & Receptionen bliver oprettet og gemt i databasen.                            \\ \hline
    Fejltilstand   & Der bliver ikke oprettet noget receptionen, og bruger 
                     får en fejl meddelelse. \\ \hline
    Aktør          & Serviceagent                                                                \\ \hline
    \end{tabular}
\end{table}

\begin{enumerate}
  \item Login med tilstrækkelig rettigheder.
  \begin{enumerate}
    \item Hvis brugeren ikke har noget login eller ikke har tilstrækkelige rettigheder. Sendes brugeren ud til login vinduet igen.
  \end{enumerate}
  \item Tryk på "ny reception" knappen.
  \item Indtast informationen den nye reception skal have.
  \item Tryk gem. Derved bliver der sendt en forespørgsel til serveren om at oprette en ny reception.
  \begin{enumerate}
    \item Hvis brugeren har indtastet ugyldig informationen eller har mistet forbindelsen til serveren, får brugeren at vide at der er sket en fejl.
  \end{enumerate}
\end{enumerate}

%\section{Ændre reception}
%\begin{enumerate}
%  \item Login med tilstrækkelig rettigheder.
%  \item Vælg den rette reception.
%  \item Tilret information.
%  \item Tryk på gem, og information bliver updateret i systemet.
%\end{enumerate}

%\section{Slet reception}
%\begin{enumerate}
%    \item Login med tilstrækkelig rettigheder.
%    \item vælg den rette reception.
%    \item Tryk på slet, hvilket sletter reception på serveren.
%\end{enumerate}

%\section{Aktivere reception}
%\begin{enumerate}
%    \item Login med tilstrækkelig rettigheder.
%    \item vælg den deaktiveret reception.
%    \item Tryk på aktivere, hvilket aktivere reception på serveren igen.
%\end{enumerate}

\begin{table}[h]
    \begin{tabular}{|p{3cm}|p{8.3cm}|}
    \hline
    Formål         & At ændre en receptions dialplan                              \\ \hline
    Succestilstand & Dialplanen er ændret og ligger klar i databasen, til at compileren henter den.                         \\ \hline
    Fejltilstand   & Der bliver ikke gemt nogen ændringer, og bruger får en fejl meddelelse. \\ \hline
    Aktør          & Serviceagent                                                                \\ \hline
    \end{tabular}
\end{table}

\section{Dialplan}
\begin{enumerate}
    \item Login med tilstrækkelig rettigheder.
  \begin{enumerate}
    \item Hvis brugeren ikke har noget login eller ikke har tilstrækkelige rettigheder. Sendes brugeren ud til login vinduet igen.
  \end{enumerate}
    \item Vælg receptionen.
    \item Begynd og redigere dens dialplan.
    \item Tryk gem og ændringerne bliver sendt til serveren der gemmer det i databasen.
  \begin{enumerate}
    \item Hvis brugeren har indtastet ugyldig informationen eller har mistet forbindelsen til serveren, får brugeren at vide at der er sket en fejl.
  \end{enumerate}
\end{enumerate}